% !TeX spellcheck = pl_PL

\rozdzial


\section{Przykładowy program demonstracyjny}
\label{roz_przyklady}

Wykonane w tej pracy rozwiązanie nie jest finalnym produktem, ale zbiorem bibliotek, modułów i oprogramowania pozwalającego na tworzenie graficznego interfejsu użytkownika. Z tego powodu konieczne stało się wykonanie przykładowej aplikacji w~celu prezentacji jego możliwości oraz ukazania praktycznego zastosowania.

Program demonstracyjny został stworzony, aby pokazać znaczną cześć funkcjonalności, która znajduje się w tej pracy. Kody źródłowe mogą posłużyć do lepszego poznania interfejsu programistycznego bibliotek. Program również steruje diodami LED i odczytuje stany przełączników umieszczonych na płytce rozwojowej. Zrzut ekranu przedstawia rysunek \ref{demoScreen}.



\begin{figure}[h]	
	\centering
	\obrazpng{0 0 861 534}{15cm}{obrazki/demoScreen.png}
	\caption{ Zrzut ekranu programu demonstracyjnego. }
	\label{demoScreen}
\end{figure}


Program pokazuje jak tworzyć i obsługiwać między innymi poniższe obiekty interfejsu użytkownika:
\begin{itemize}
	\item Pasek zakładek pozwalający wybrać aktualnie wyświetlaną cześć interfejsu użytkownika,
	\item Pasek statusu wyświetlający informacje dla użytkownika na dole okna,
	\item Wyskakujące okno z wiadomością pozwalające na informowanie lub zapytanie użytkownika,
	\item Prosty wykres zawierający zmieniające się dane w sposób ciągły,
	\item Kontrolki opisane w rozdziale \ref{Przykladowe_obiekty}, tzn. przycisk, tabela, pole wielokrotnego wyboru, pole pojedynczego wyboru, etykieta, lista wyboru, pole tekstowe, pasek postępu.
\end{itemize}

